\documentclass[12pt]{article}

\usepackage{fouriernc}
\usepackage{amsmath}

\begin{document}

\section*{Problem-120}
\subsection*{Problem Statement}
Can you factor any other polynomial of the form $a^{2n} + b^{2n}$?

\subsection*{Solution}
We shall consider two separate cases: (1) when $n$ is even (2) when $n$ is odd.

\subsubsection*{When $n$ is even, therefore $n$ has the form $2m$ with $m \geq 1$}

\begin{equation*}
	\begin{aligned}
		a^{2n}+b^{2n} &= a^{4m}+b^{4m}\\
					  &= \left( a^{2m} \right)^2 + \left( b^{2m} \right)^2\\
					  &= \left( a^{2m} \right)^2 + 2\cdot a^{2m}\cdot b^{2m}+ \left( b^{2m} \right)^2 - 2\cdot a^{2m}\cdot b^{2m}\\
					  &= \left( a^{2m}+b^{2m} \right)^2 - \left( \sqrt{2} \cdot a^m \cdot b^m \right)^2\\
					  &= \left( a^{2m}+ \sqrt{2} \cdot a^m \cdot b^m + b^{2m} \right)\ \left( a^{2m} - \sqrt{2} \cdot a^m \cdot b^m + b^{2m} \right)
	\end{aligned}
\end{equation*}

\subsubsection*{When $n$ is odd, therefore $n$ has the form $2m+1$ with $m \geq 0$}

Let's consider some examples and try to generalize from them.
\begin{itemize}
\item When $m=0$ therefore $n=1$.
\begin{equation*}
	\begin{aligned}
		a^2+b^2 &= a^2 - \left( -b^2 \right)\\
				&= a^2 - \left( \sqrt{-1}\cdot b \right)^2\\
				&= a^2 - (i\cdot b)^2\\
				&= (a+i\cdot b)\ (a-i\cdot b)
	\end{aligned}
\end{equation*}

\item When $m=1$ therefore $n=3$.
Our polynomial in this case is $a^{6}+b^{6}$. We notice that if we set $a^2 = -b^2$, the polynomial evaluates to zero. So, we might be able to factor the polynomial into $\left(a^2+b^2\right)\ (\ \ldots\ )$. We thus try to extract $a^2+b^2$ from each pair of consecutive terms, introducing temporary terms as required.
\begin{equation*}
	\begin{aligned}
		a^6+b^6 &= a^6 + a^4\cdot b^2 - a^4\cdot b^2 - a^2\cdot b^4 + a^2\cdot b^4 + b^6\\
				&= a^4\cdot \left( a^2+b^2 \right) - a^2\cdot b^2\cdot \left( a^2+b^2 \right) + b^4\cdot \left( a^2+b^2 \right)\\
				&= \left( a^2+b^2 \right)\ \left( a^4-a^2\cdot b^2+b^4 \right)
	\end{aligned}
\end{equation*}
\item When $m=2$ therefore $n=5$.
Our polynomial in this case is $a^{10}+b^{10}$. Again, setting $a^2=-b^2$ makes the polynomial zero. So, $a^2+b^2$ is a potential factor. We again try to extract $a^2+b^2$ from each pair of consecutive terms introducing temporary terms as required.
\begin{equation*}
	\begin{aligned}
		& a^{10} + b^{10}\\
		&=\ a^{10} + a^8\cdot b^2 - a^8\cdot b^2 - a^6\cdot b^4 + a^6\cdot b^4 + a^4\cdot b^6 - a^4\cdot b^6 - a^2\cdot b^8 + a^2\cdot b^8 + b^{10}\\
		&= a^8 \cdot \left( a^2+b^2 \right) - a^6\cdot b^2\cdot \left( a^2+b^2\right) + a^4\cdot b^4\cdot \left( a^2+b^2 \right) - a^2\cdot b^6\cdot \left( a^2+b^2 \right) + b^8\cdot \left( a^2+b^2 \right)\\
		&= \left( a^2+b^2 \right)\ \left( a^8 - a^6\cdot b^2 + a^4\cdot b^4 - a^2\cdot b^6 + b^8 \right)
	\end{aligned}
\end{equation*}
\end{itemize}

The factoring process for $n > 1$ looks like below:
\begin{equation*}
	\begin{aligned}
	a^{2n}+b^{2n} &= \sum_{k=0}^{n-1}\ (-1)^k\ \left[ a^{2(n-k)} \cdot b^{2k}\ +\ a^{2(n-k-1)}\cdot b^{2(k+1)} \right]\\
				  &= \sum_{k=0}^{n-1}\ (-1)^k\ a^{2(n-k-1)}\cdot b^{2k}\ \left[ a^2+b^2 \right]\\
				  &= \left( a^2+b^2 \right)\ \left( \sum_{k=0}^{n-1}\ (-1)^k\ a^{2(n-k-1)}\cdot b^{2k} \right)
	\end{aligned}
\end{equation*}

From the above three cases, we can generalize the factoring of $a^{2n}+b^{2n}$ when $n$ is odd as follows:
\[
a^{2n}+b^{2n} = \begin{cases} (a+i\cdot b)\ (a-i\cdot b) &\mbox{, when } n=1\\ \left( a^2+b^2 \right)\ \left( \sum_{k=0}^{n-1}\ (-1)^k\ a^{2(n-k-1)}\cdot b^{2k} \right) & \mbox{, otherwise} \end{cases}
\]

\section*{Problem-122}


\end{document}



















