\documentclass[12pt]{article}

\usepackage{fouriernc}
\usepackage{amsmath}
\usepackage{enumitem} % alphabetic enumerate

\begin{document}

\section*{Problem-120}
\subsection*{Problem Statement}
Can you factor any other polynomial of the form $a^{2n} + b^{2n}$?

\subsection*{Solution}
From the definition of a polynomial, $n \geq 0$. We shall consider two separate cases: (1) when $n$ is even (2) when $n$ is odd.

\subsubsection*{When $n$ is even, therefore $n$ has the form $2m$ with $m \geq 0$}

\begin{equation*}
	\begin{aligned}
		a^{2n}+b^{2n} &= a^{4m}+b^{4m}\\
					  &= \left( a^{2m} \right)^2 + \left( b^{2m} \right)^2\\
					  &= \left( a^{2m} \right)^2 + 2\cdot a^{2m}\cdot b^{2m}+ \left( b^{2m} \right)^2 - 2\cdot a^{2m}\cdot b^{2m}\\
					  &= \left( a^{2m}+b^{2m} \right)^2 - \left( \sqrt{2} \cdot a^m \cdot b^m \right)^2\\
					  &= \left( a^{2m}+ \sqrt{2} \cdot a^m \cdot b^m + b^{2m} \right)\ \left( a^{2m} - \sqrt{2} \cdot a^m \cdot b^m + b^{2m} \right)
	\end{aligned}
\end{equation*}

\subsubsection*{When $n$ is odd, therefore $n$ has the form $2m+1$ with $m \geq 0$}

Let's consider some examples and try to generalize from them.
\begin{itemize}
\item When $m=0$ therefore $n=1$.
\begin{equation*}
	\begin{aligned}
		a^2+b^2 &= a^2 - \left( -b^2 \right)\\
				&= a^2 - \left( \sqrt{-1}\cdot b \right)^2\\
				&= a^2 - (i\cdot b)^2\\
				&= (a+i\cdot b)\ (a-i\cdot b)
	\end{aligned}
\end{equation*}

\item When $m=1$ therefore $n=3$.
Our polynomial in this case is $a^{6}+b^{6}$. We notice that if we set $a^2 = -b^2$, the polynomial evaluates to zero. So, we might be able to factor the polynomial into $\left(a^2+b^2\right)\ (\ \ldots\ )$. We thus try to extract $a^2+b^2$ from each pair of consecutive terms, introducing temporary terms as necessary.
\begin{equation*}
	\begin{aligned}
		a^6+b^6 &= a^6 + a^4\cdot b^2 - a^4\cdot b^2 - a^2\cdot b^4 + a^2\cdot b^4 + b^6\\
				&= a^4\cdot \left( a^2+b^2 \right) - a^2\cdot b^2\cdot \left( a^2+b^2 \right) + b^4\cdot \left( a^2+b^2 \right)\\
				&= \left( a^2+b^2 \right)\ \left( a^4-a^2\cdot b^2+b^4 \right)
	\end{aligned}
\end{equation*}
\item When $m=2$ therefore $n=5$.
Our polynomial in this case is $a^{10}+b^{10}$. Again, setting $a^2=-b^2$ makes the polynomial zero. So, $a^2+b^2$ is a potential factor. We again try to extract $a^2+b^2$ from each pair of consecutive terms introducing temporary terms as necessary.
\begin{equation*}
	\begin{aligned}
		& a^{10} + b^{10}\\
		&=\ a^{10} + a^8\cdot b^2 - a^8\cdot b^2 - a^6\cdot b^4 + a^6\cdot b^4 + a^4\cdot b^6 - a^4\cdot b^6 - a^2\cdot b^8 + a^2\cdot b^8 + b^{10}\\
		&= a^8 \cdot \left( a^2+b^2 \right) - a^6\cdot b^2\cdot \left( a^2+b^2\right) + a^4\cdot b^4\cdot \left( a^2+b^2 \right) - a^2\cdot b^6\cdot \left( a^2+b^2 \right) + b^8\cdot \left( a^2+b^2 \right)\\
		&= \left( a^2+b^2 \right)\ \left( a^8 - a^6\cdot b^2 + a^4\cdot b^4 - a^2\cdot b^6 + b^8 \right)
	\end{aligned}
\end{equation*}
\end{itemize}

The factoring process for $n > 1$ looks like below:
\begin{equation*}
	\begin{aligned}
	a^{2n}+b^{2n} &= \sum_{k=0}^{n-1}\ (-1)^k\ \left[ a^{2(n-k)} \cdot b^{2k}\ +\ a^{2(n-k-1)}\cdot b^{2(k+1)} \right]\\
				  &= \sum_{k=0}^{n-1}\ (-1)^k\ a^{2(n-k-1)}\cdot b^{2k}\ \left[ a^2+b^2 \right]\\
				  &= \left( a^2+b^2 \right)\ \left( \sum_{k=0}^{n-1}\ (-1)^k\ a^{2(n-k-1)}\cdot b^{2k} \right)
	\end{aligned}
\end{equation*}

From the above three cases, we can generalize the factoring of $a^{2n}+b^{2n}$ when $n$ is odd as follows:
\[
a^{2n}+b^{2n} = \begin{cases} (a+i\cdot b)\ (a-i\cdot b) &\mbox{, when } n=1\\ \left( a^2+b^2 \right)\ \left( \sum_{k=0}^{n-1}\ (-1)^k\ a^{2(n-k-1)}\cdot b^{2k} \right) & \mbox{, otherwise} \end{cases}
\]
\pagebreak
\section*{Problem-122(c)}
\subsection*{Problem Statement}
Factor $a^{10} + a^5 + 1$.
\subsection*{Solution}
\begin{equation*}
	\begin{aligned}
			& a^{10} + a^5 + 1\\
			&= \left( a^5 \right)^2 + 2 \cdot a^5 \cdot \frac{1}{2} + \left( \frac{1}{2} \right)^2 + \frac{3}{4}\\
			&= \left( a^5+\frac{1}{4} \right)^2 + \left( \frac{ \sqrt{3} }{2} \right)^2\\
			&= \left( a^5+\frac{1}{4} \right)^2 - \left( i\cdot \frac{ \sqrt{3} }{2} \right)^2\\
			&= \left( a^5 + \frac{1}{4} + \frac{i\cdot \sqrt{3}}{2} \right)\ \left( a^5 + \frac{1}{4} - \frac{i\cdot \sqrt{3}}{2} \right)\\
			&= \left( a^5 + \frac{1 + 2\sqrt{3}\cdot i}{4} \right)\ \left( a^5 + \frac{1-2\sqrt{3}\cdot i}{4} \right)
	\end{aligned}
\end{equation*}

\section*{Problem-122(d)}
\subsection*{Problem Statement}
Factor $a^3 + b^3 + c^3 -3abc$.
\subsection*{Solution}
We note that setting $a=b=c$ makes the polynomial zero. A possible factor may be $a^2+b^2+c^2-ab-bc-ca$. So, from each set of consecutive six terms we shall extract out this potential factor, introducing temporary terms as necessary.
\begin{equation*}
	\begin{aligned}
		& a^3 + b^3 + c^3 -3abc\\
		&=\ \left(a^3+ab^2+ac^2-a^2b-abc-ca^2\right) + \left( a^2b+b^3+bc^2-ab^2-b^2c-abc \right) \\
		&+ \left( ca^2 + b^2c + c^3 - abc - bc^2 - c^2a \right)\\
		&= a\cdot \left( a^2+b^2+c^2-ab-bc-ca \right) + b\cdot \left( a^2+b^2+c^2-ab-bc-ca \right)\\
		&+\ c\cdot \left( a^2+b^2+c^2-ab-bc-ca \right)\\
		&= (a+b+c)\ \left( a^2+b^2+c^2-ab-bc-ca \right)
	\end{aligned}
\end{equation*}
Yep, it looks like magic, pulling things out of thin air to make it work. But this is expected, since when we multiply factors of a polynomial (the reverse of factoring), we collect the similar terms and in that process some terms may vanish. It reminds me that differentiation is straightforward, integration is hard. Here, multiplication of polynomials is easy, factoring a polynomial is hard.

\section*{Problem-122(e)}
\subsection*{Problem Statement}
Factor $(a+b+c)^3 -a^3-b^3-c^3$.
\subsection*{Solution}
We note that setting $a=-b$ makes the polynomial zero. So, $a+b$ is a potential factor.
\begin{equation*}
	\begin{aligned}
		& (a+b+c)^3 -a^3-b^3-c^3\\
		&=\ \left( 3a^2b+3ab^2 \right) + \left( 3abc+3b^2c \right) + \left( 3a^2c+3abc \right) + \left( 3ac^2 + 3bc^2 \right)\\
		&= 3ab\cdot (a+b) + 3bc\cdot (a+b) + 3ca\cdot (a+b) + 3c^2(a+b)\\
		&= 3(a+b)\ \left(ab+bc+ca+c^2\right)\\
		&= 3(a+b)\ [\ b(a+c)+c(a+c)\ ]\\
		&= 3\ (a+b)\ (b+c)\ (c+a)
	\end{aligned}
\end{equation*}
Actually there is nothing special about the pair $(a,b)$, all three pairs behave identically. We could set $b=-c$ or $c=-a$ to make the polynomial zero. Hence all three $a+b$, $b+c$, and $c+a$ are factors.

\section*{Problem-122(f)}
\subsection*{Problem Statement}
Factor $(a-b)^3 + (b-c)^3 + (c-a)^3$.
\subsection*{Solution}
We note that setting $a=b$ or $b=c$ or $c=a$ makes the polynomial zero. So, possibly the polynomial has the form $k\cdot (a-b)\ (b-c)\ (c-a)$ where $k$ is some constant. Following similar procedure as in $122(e)$ we find below factors:
\[
	(a-b)^3 + (b-c)^3 + (c-a)^3 = 3\ (a-b)\ (b-c)\ (c-a)
\]

\section*{Problem-134}
\subsection*{Problem Statement}
You know that $x + \frac{1}{x}$ is an integer. Prove that $x^n + \frac{1}{x^n}$ is an integer for any $n=1,2,3,$ etc.

\subsection*{Solution}
Let's work with some examples and try to generalize from them.
\begin{itemize}
\item When $n=2$, we need to show $x^2 + \frac{1}{x^2}$ is an integer.
\begin{equation*}
	\begin{aligned}
		\left(x + \frac{1}{x}\right)^2 &= x^2 + 2 \cdot x \cdot \frac{1}{x} + \frac{1}{x^2}\\
		&= x^2+2+\frac{1}{x^2}
	\end{aligned}
\end{equation*}
So, we have $x^2+\frac{1}{x^2} = \left(x + \frac{1}{x}\right)^2-2$. Since $x + \frac{1}{x}$ is an integer, its square is also an integer. And, if we subtract $2$ from the square, the result will still be an integer.

\item When $n=3$, we need to show $x^3 + \frac{1}{x^3}$ is an integer.
\begin{equation*}
	\begin{aligned}
		\left(x + \frac{1}{x}\right)^3 &= x^3 + 3 \cdot x^2 \cdot \frac{1}{x} + 3 \cdot x \cdot \frac{1}{x^2} + \frac{1}{x^3}\\
		&= x^3+\frac{1}{x^3} + 3\cdot \left( x + \frac{1}{x} \right)
	\end{aligned}
\end{equation*}
We have $x^3 + \frac{1}{x^3} = \left(x + \frac{1}{x}\right)^3-3\cdot \left( x + \frac{1}{x} \right)$, which is an integer.

\item When $n=4$, we need to show $x^4 + \frac{1}{x^4}$ is an integer. Using binomial theorem we can expand $\left(x + \frac{1}{x}\right)^4$ as follows:
\begin{equation*}
	\begin{aligned}
		\left(x + \frac{1}{x}\right)^4 &= \sum_{k=0}^{4} \binom{4}{k}\ x^{4-k}\ \left( \frac{1}{x} \right)^k\\
		&= x^4 + \binom{4}{1}x^3 \cdot \frac{1}{x} + \binom{4}{2}x^2\ \cdot \frac{1}{x^2} + \binom{4}{3}x \cdot \frac{1}{x^3} + \frac{1}{x^4}\\
		&= x^4 + \frac{1}{x^4} + \binom{4}{1}\ \left( x^2 + \frac{1}{x^2} \right) + \binom{4}{2}\ \ \ \ \ \ \textrm{// Using}\ \binom{n}{k} = \binom{n}{n-k}
	\end{aligned}
\end{equation*}
We have already shown that $x^2+\frac{1}{x^2}$ is an integer. 

So $x^4 + \frac{1}{x^4} = \left(x + \frac{1}{x}\right)^4 - \binom{4}{1}\ \left( x^2 + \frac{1}{x^2} \right)- \binom{4}{2}$ is an integer.

\item When $n=5$.
\begin{equation*}
	\begin{aligned}
		\left( x + \frac{1}{x} \right)^5 &= \sum_{k=0}^5 \binom{5}{k} x^{5-k} \frac{1}{x^k}\\
		&= x^5 + \frac{1}{x^5} + \binom{5}{1}\left( x^3 + \frac{1}{x^3} \right) + \binom{5}{2} \left( x + \frac{1}{x} \right)
	\end{aligned}
\end{equation*}
\end{itemize}
From the above examples, we see that if we expand $\left(x + \frac{1}{x}\right)^n$ we get terms like $x^m + \frac{1}{x^m}$ where $m \leq n$. This suggests using strong induction. To show $x^n + \frac{1}{x^n}$ is an integer, let's expand $\left(x+\frac{1}{x}\right)^n$.
\begin{equation*}
	\begin{aligned}
		& \left(x+\frac{1}{x}\right)^n\\
		&= \sum_{k=0}^n \binom{n}{k} x^{n-k} \frac{1}{x^{k}}\\
		&= x^n + \frac{1}{x^n} + \sum_{k=1}^{n-1}\binom{n}{k}x^{n-k}\frac{1}{x^k}
	\end{aligned}
\end{equation*}
When $n$ is odd, we can pair up the terms and write as follows:
\[
	\left(x+\frac{1}{x}\right)^n = x^n + \frac{1}{x^n} + \sum_{k=1}^{\frac{n-1}{2}}\binom{n}{k}\left( x^{n-2k} + \frac{1}{x^{n-2k}} \right)
\]
By strong induction we know $x^{n-2k} + \frac{1}{x^{n-2k}}$ are integers, thus $x^n + \frac{1}{x^n}$ is an integer.

When $n$ is even, there are odd number of terms in the binomial expansion, with a lone center at $k = n-k$ which gives an integer $\binom{n}{n/2}$. Thus, for even $n$ we may write:
\[
	\left(x+\frac{1}{x}\right)^n = x^n + \frac{1}{x^n} + \sum_{k=1}^{\frac{n-1}{2}}\binom{n}{k}\left( x^{n-2k} + \frac{1}{x^{n-2k}} \right) + \binom{n}{n/2}
\]

Again, by strong induction we know $x^{n-2k} + \frac{1}{x^{n-2k}}$ are integers, thus $x^n + \frac{1}{x^n}$ is an integer.  $\blacksquare$

\section*{Problem-153}
\subsection*{Problem Statement}
The polynomial $P(x) = x^3 + x^2 -10x + 1$ has three different roots (the authors guarantee it) denoted by $x_1, x_2, x_3$. Write a polynomial with integer coefficients having roots
\begin{enumerate}[label=(\alph*)]
	\item $x_1+1, x_2+1, x_3+1$
	\item $2x_1, 2x_2, 2x_3$
	\item $\frac{1}{x_1}, \frac{1}{x_2}, \frac{1}{x_3}$
\end{enumerate}

\subsection*{Polynomial with roots $x_1+1, x_2+1, x_3+1$}
Since $x_1$ is a root of $P(x)$, $P(x_1) = 0$. Thus, we want to find a polynomial $S(x)$ such that $S(x_1+1) = P(x_1)$ for example.
\begin{equation*}
	\begin{aligned}
		S(x) &= (x-1)^3 + (x-1)^2 - 10\ (x-1) + 1\\
			&= x^3 - 2x^2-9x+11
	\end{aligned}
\end{equation*}

\subsection*{Polynomial with roots $2x_1, 2x_2, 2x_3$}
A good start would be with a polynomial say $T(x)$ such that $T(2x_1) = P(x_1)$.
\begin{equation*}
	\begin{aligned}
		T(x) &= \left( \frac{x}{2} \right)^3 + \left( \frac{x}{2} \right)^2 - 10\ \left( \frac{x}{2} \right) + 1\\
		&= \frac{1}{8}x^3 + \frac{1}{4}x^2-5x+1
	\end{aligned}
\end{equation*}
$T(x)$ is close but not the solution, since we need integer coefficients. So, we take $S(x)$, a constant multiple of $T(x)$, as our solution.
\[
	S(x) = 8\ T(x) = x^3 + 2x^2 -40x + 8
\]

\subsection*{Polynomial with roots $\frac{1}{x_1}, \frac{1}{x_2}, \frac{1}{x_3}$}
Again, a good start would be with a polynomial say $T(x)$ such that 

$T\left( \frac{1}{x_1} \right) = P(x_1)$. Consider $W(x)$ below:

\begin{equation*}
	\begin{aligned}
		W(x) = \frac{1}{x^3} + \frac{1}{x^2} - 10 \cdot \frac{1}{x} + 1
	\end{aligned}
\end{equation*}
Here $W(x)$ is not really a polynomial, since negative power is not allowed in a polynomial by definition. However, $W(x)$ has the property that 

$W\left( \frac{1}{x_1} \right) = P(x_1)$.

To fix that, we can take $S(x) = x^3 \cdot W(x)$ as our solution. Because in that case $S\left( \frac{1}{x_1} \right) = \frac{1}{x_1^3} \cdot W\left( \frac{1}{x_1} \right) = \frac{1}{x_1^3} \cdot P(x_1) = 0$.
\[
	S(x) = x^3 \cdot W(x) = x^3 - 10x^2 + x + 1
\]

\section*{Problem-154}
\subsection*{Problem Statement}
Assume that $x^3 + ax^2 + x + b$ (where $a$ and $b$ are some numbers) is divisible by $x^2-3x+2$. Find $a$ and $b$.

\subsection*{Solution}
Since $P(x) = x^3 + ax^2 + x + b$ is divisible by $x^2-3x+2$, we have:
\begin{equation*}
	\begin{aligned}
		P(x) &= \left( x^2-3x+2 \right) \cdot Q(x)\\
						   &= (x-1) \ (x-2) \ Q(x)
	\end{aligned}
\end{equation*}
So, $1$ and $2$ are roots of $P(x)$.
\begin{equation}
	P(1) = a+b+2 = 0
\end{equation}
\begin{equation}
	P(2) = 4a + b + 10 = 0
\end{equation}
\\
Solving $(1)$ and $(2)$ gives us: $(a, b) = \left( -\frac{8}{3}, \frac{2}{3} \right)$.

\section*{Problem-156}
\subsection*{Problem Statement}
Let us write the values of $P(0),\ P(1),\ P(2),\ \ldots$ for $P(x) = x^2-x-4$.
\[
	-4,\ -4,\ -2,\ 2,\ 8,\ 16,\ 26,\ \ldots
\]
Under any two adjacent numbers write their difference:
\[
\begin{array}{lllllllllllll}
	-4 &   & -4 &   & -2 &   & 2 &   & 8 &   & 16 &    & 26\ \ldots\\
	   & 0 &    & 2 &    & 4 &   & 6 &   & 8 &    & 10\ \ldots &
\end{array}
\]
and repeat the same operation with this sequence of ``first differences'':
\[
\begin{array}{lllllllllllll}
	-4 &   & -4 &   & -2 &   & 2 &   & 8 &   & 16 &    & 26\ \dots\\
	   & 0 &    & 2 &    & 4 &   & 6 &   & 8 &    & 10\ \ldots &   \\
	   &   & \ \ 2 &   &\ \ 2 &   & 2 &   & 2 &   &\ \ 2\ \ldots &    &
\end{array}
\]
Now all numbers are $2$ s. Prove that it is not a coincidence and that all subsequent numbers (called ``second differences'') are also $2$ s.
\subsection*{Solution}
If $P_{\Delta}(x)$ represents ``first differences'', we have:
\begin{equation*}
	\begin{aligned}
		P_{\Delta}(x) &= P(x+1)-P(x)\\
					  &= 2x
	\end{aligned}
\end{equation*}
Let $P_{\Delta \Delta}(x)$ denote the ``second differences''.
\begin{equation*}
	\begin{aligned}
		P_{\Delta \Delta} &= P_{\Delta}(x+1)-P_{\Delta}(x)\\
		                  &= 2
	\end{aligned}
\end{equation*}

\end{document}



















