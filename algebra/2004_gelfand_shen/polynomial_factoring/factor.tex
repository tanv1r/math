\documentclass[12pt]{article}

\usepackage{fouriernc}
\usepackage{amsmath}

\begin{document}

\section*{Problem-120}
\subsection*{Problem Statement}
Can you factor any other polynomial of the form $a^{2n} + b^{2n}$?

\subsection*{Solution}
From the definition of a polynomial, $n \geq 0$. We shall consider two separate cases: (1) when $n$ is even (2) when $n$ is odd.

\subsubsection*{When $n$ is even, therefore $n$ has the form $2m$ with $m \geq 0$}

\begin{equation*}
	\begin{aligned}
		a^{2n}+b^{2n} &= a^{4m}+b^{4m}\\
					  &= \left( a^{2m} \right)^2 + \left( b^{2m} \right)^2\\
					  &= \left( a^{2m} \right)^2 + 2\cdot a^{2m}\cdot b^{2m}+ \left( b^{2m} \right)^2 - 2\cdot a^{2m}\cdot b^{2m}\\
					  &= \left( a^{2m}+b^{2m} \right)^2 - \left( \sqrt{2} \cdot a^m \cdot b^m \right)^2\\
					  &= \left( a^{2m}+ \sqrt{2} \cdot a^m \cdot b^m + b^{2m} \right)\ \left( a^{2m} - \sqrt{2} \cdot a^m \cdot b^m + b^{2m} \right)
	\end{aligned}
\end{equation*}

\subsubsection*{When $n$ is odd, therefore $n$ has the form $2m+1$ with $m \geq 0$}

Let's consider some examples and try to generalize from them.
\begin{itemize}
\item When $m=0$ therefore $n=1$.
\begin{equation*}
	\begin{aligned}
		a^2+b^2 &= a^2 - \left( -b^2 \right)\\
				&= a^2 - \left( \sqrt{-1}\cdot b \right)^2\\
				&= a^2 - (i\cdot b)^2\\
				&= (a+i\cdot b)\ (a-i\cdot b)
	\end{aligned}
\end{equation*}

\item When $m=1$ therefore $n=3$.
Our polynomial in this case is $a^{6}+b^{6}$. We notice that if we set $a^2 = -b^2$, the polynomial evaluates to zero. So, we might be able to factor the polynomial into $\left(a^2+b^2\right)\ (\ \ldots\ )$. We thus try to extract $a^2+b^2$ from each pair of consecutive terms, introducing temporary terms as necessary.
\begin{equation*}
	\begin{aligned}
		a^6+b^6 &= a^6 + a^4\cdot b^2 - a^4\cdot b^2 - a^2\cdot b^4 + a^2\cdot b^4 + b^6\\
				&= a^4\cdot \left( a^2+b^2 \right) - a^2\cdot b^2\cdot \left( a^2+b^2 \right) + b^4\cdot \left( a^2+b^2 \right)\\
				&= \left( a^2+b^2 \right)\ \left( a^4-a^2\cdot b^2+b^4 \right)
	\end{aligned}
\end{equation*}
\item When $m=2$ therefore $n=5$.
Our polynomial in this case is $a^{10}+b^{10}$. Again, setting $a^2=-b^2$ makes the polynomial zero. So, $a^2+b^2$ is a potential factor. We again try to extract $a^2+b^2$ from each pair of consecutive terms introducing temporary terms as necessary.
\begin{equation*}
	\begin{aligned}
		& a^{10} + b^{10}\\
		&=\ a^{10} + a^8\cdot b^2 - a^8\cdot b^2 - a^6\cdot b^4 + a^6\cdot b^4 + a^4\cdot b^6 - a^4\cdot b^6 - a^2\cdot b^8 + a^2\cdot b^8 + b^{10}\\
		&= a^8 \cdot \left( a^2+b^2 \right) - a^6\cdot b^2\cdot \left( a^2+b^2\right) + a^4\cdot b^4\cdot \left( a^2+b^2 \right) - a^2\cdot b^6\cdot \left( a^2+b^2 \right) + b^8\cdot \left( a^2+b^2 \right)\\
		&= \left( a^2+b^2 \right)\ \left( a^8 - a^6\cdot b^2 + a^4\cdot b^4 - a^2\cdot b^6 + b^8 \right)
	\end{aligned}
\end{equation*}
\end{itemize}

The factoring process for $n > 1$ looks like below:
\begin{equation*}
	\begin{aligned}
	a^{2n}+b^{2n} &= \sum_{k=0}^{n-1}\ (-1)^k\ \left[ a^{2(n-k)} \cdot b^{2k}\ +\ a^{2(n-k-1)}\cdot b^{2(k+1)} \right]\\
				  &= \sum_{k=0}^{n-1}\ (-1)^k\ a^{2(n-k-1)}\cdot b^{2k}\ \left[ a^2+b^2 \right]\\
				  &= \left( a^2+b^2 \right)\ \left( \sum_{k=0}^{n-1}\ (-1)^k\ a^{2(n-k-1)}\cdot b^{2k} \right)
	\end{aligned}
\end{equation*}

From the above three cases, we can generalize the factoring of $a^{2n}+b^{2n}$ when $n$ is odd as follows:
\[
a^{2n}+b^{2n} = \begin{cases} (a+i\cdot b)\ (a-i\cdot b) &\mbox{, when } n=1\\ \left( a^2+b^2 \right)\ \left( \sum_{k=0}^{n-1}\ (-1)^k\ a^{2(n-k-1)}\cdot b^{2k} \right) & \mbox{, otherwise} \end{cases}
\]
\pagebreak
\section*{Problem-122(c)}
\subsection*{Problem Statement}
Factor $a^{10} + a^5 + 1$.
\subsection*{Solution}
\begin{equation*}
	\begin{aligned}
			& a^{10} + a^5 + 1\\
			&= \left( a^5 \right)^2 + 2 \cdot a^5 \cdot \frac{1}{2} + \left( \frac{1}{2} \right)^2 + \frac{3}{4}\\
			&= \left( a^5+\frac{1}{4} \right)^2 + \left( \frac{ \sqrt{3} }{2} \right)^2\\
			&= \left( a^5+\frac{1}{4} \right)^2 - \left( i\cdot \frac{ \sqrt{3} }{2} \right)^2\\
			&= \left( a^5 + \frac{i\cdot \sqrt{3}}{2} + \frac{1}{4} \right)\ \left( a^5 - \frac{i\cdot \sqrt{3}}{2} + \frac{1}{4} \right)\\
			&= \left( a^5 + \frac{1 + 2\sqrt{3}\cdot i}{4} \right)\ \left( a^5 - \frac{1-2\sqrt{3}\cdot i}{4} \right)
	\end{aligned}
\end{equation*}

\section*{Problem-122(d)}
\subsection*{Problem Statement}
Factor $a^3 + b^3 + c^3 -3abc$.
\subsection*{Solution}
We note that setting $a=b=c$ makes the polynomial zero. A possible factor may be $a^2+b^2+c^2-ab-bc-ca$. So, from each set of consecutive six terms we shall extract out this potential factor, introducing temporary terms as necessary.
\begin{equation*}
	\begin{aligned}
		& a^3 + b^3 + c^3 -3abc\\
		&=\ \left(a^3+ab^2+ac^2-a^2b-abc-ca^2\right) + \left( a^2b+b^3+bc^2-ab^2-b^2c-abc \right) \\
		&+ \left( ca^2 + b^2c + c^3 - abc - bc^2 - c^2a \right)\\
		&= a\cdot \left( a^2+b^2+c^2-ab-bc-ca \right) + b\cdot \left( a^2+b^2+c^2-ab-bc-ca \right)\\
		&+\ c\cdot \left( a^2+b^2+c^2-ab-bc-ca \right)\\
		&= (a+b+c)\ \left( a^2+b^2+c^2-ab-bc-ca \right)
	\end{aligned}
\end{equation*}
Yep, it looks like magic, pulling things out of thin air to make it work. But this is expected, since when we multiply factors of a polynomial (the reverse of factoring), we collect the similar terms and in that process some terms may vanish. It reminds me that differentiation is straightforward, integeration is hard. Here, multiplication of polynomials is easy, factoring a polynomial is hard.

\section*{Problem-122(e)}
\subsection*{Problem Statement}
Factor $(a+b+c)^3 -a^3-b^3-c^3$.
\subsection*{Solution}
We note that setting $a=-b$ makes the polynomial zero. So, $a+b$ is a potential factor.
\begin{equation*}
	\begin{aligned}
		& (a+b+c)^3 -a^3-b^3-c^3\\
		&=\ \left( 3a^2b+3ab^2 \right) + \left( 3abc+3b^2c \right) + \left( 3a^2c+3abc \right) + \left( 3ac^2 + 3bc^2 \right)\\
		&= 3ab\cdot (a+b) + 3bc\cdot (a+b) + 3ca\cdot (a+b) + 3c^2(a+b)\\
		&= 3(a+b)\ \left(ab+bc+ca+c^2\right)\\
		&= 3(a+b)\ [\ b(a+c)+c(a+c)\ ]\\
		&= 3\ (a+b)\ (b+c)\ (c+a)
	\end{aligned}
\end{equation*}
Actually there is nothing special about the pair $(a,b)$, all three pairs behave identically. We could set $b=-c$ or $c=-a$ to make the polynomial zero. Hence all three $a+b$, $b+c$, and $c+a$ are factors.

\section*{Problem-122(f)}
\subsection*{Problem Statement}
Factor $(a-b)^3 + (b-c)^3 + (c-a)^3$.
\subsection*{Solution}
We note that setting $a=b$ or $b=c$ or $c=a$ makes the polynomial zero. So, possibly the polynomial has the form $k\cdot (a-b)\ (b-c)\ (c-a)$ where $k$ is some constant. Following similar procedure as in $122(e)$ we find below factors:
\[
	(a-b)^3 + (b-c)^3 + (c-a)^3 = 3\ (a-b)\ (b-c)\ (c-a)
\]

\end{document}



















