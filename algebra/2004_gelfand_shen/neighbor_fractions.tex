\documentclass{article}

\usepackage{fouriernc}
\usepackage{enumerate} %Needed to have a, b, c instead of default 1, 2, 3 in enumerate environment%

\usepackage{amsmath}

\begin{document}

\section*{Problem-42}

\subsection*{Problem Statement}
Fractions $\frac{a}{b}$ and $\frac{c}{d}$ are called neighbor fractions if their difference $\frac{ad-bc}{bd}$ has numerator $\pm 1$, that is, $ad-bc = \pm 1$. Prove that 
\begin{enumerate}[a.]
\item in this case neither fraction can be simplified (that is, neither has any common factors in numerator and denominator);
\item if $\frac{a}{b}$ and $\frac{c}{d}$ are neighbor fractions, then $\frac{a+c}{b+d}$ is between them and is a neighbor fraction for both $\frac{a}{b}$ and $\frac{c}{d}$; moreover,
\item no fraction $\frac{e}{f}$ with positive integer $e$ and $f$ such that $f < b+d$ is between $\frac{a}{b}$ and $\frac{c}{d}$.

(In part b. of the statement the book says $\frac{a+b}{c+d}$ instead of $\frac{a+c}{b+d}$ and looks like that is a typo. For, neighbor fractions $\frac{1}{3}$ and $\frac{1}{2}$ the composite fraction $\frac{1+3}{1+2}$ is not in-between.)
\end{enumerate}
\subsection*{Solution to Part a.}
Looks like for neighbor fractions $\frac{0}{1}$ and $\frac{1}{1}$ this claim does not hold. So, I shall assume that $a, b, c, d$ are all positive integers, $a \neq b, c \neq d$, and if a common factor exists, it is greater than $1$. In other words, if the only common factor between say $a$ and $b$ is $1$, I consider them not having a common factor at all.

We shall use \textbf{proof by contradiction}, therefore we shall assume that two neighbor fractions share a common factor and that assumption will lead to a false conclusion, completing the proof. 

Say $a < b$, $a$ and $b$ share a common factor $f$, and $f$ is greater than $1$. Since the common factor is greater than $1$, $a > 1$. So, we have $b = f\cdot a$.

From the definition of neighbor fraction given in the problem statement, we have:

\begin{displaymath}
\begin{split}
	ad - bc         & = \pm 1\\
	ad-(f \cdot a)c & = \pm 1\\
	a(d-fc)         & = \pm 1\\
	d-fc            & = \pm \frac{1}{a}
\end{split}
\end{displaymath}
Since we started with integers on both sides of the equation and ended up with an integer on the left and a proper fraction on the right, we have a contradiction. Similar idea works if we assumed $a > b$ or if we assumed $c$ and $d$ also share a common factor. $\blacksquare$

\subsection*{Solution to Part b.}
Say $\frac{a}{b} > \frac{c}{d}$. Let's now interpret the neighbor fractions $\frac{a}{b}$ and $\frac{c}{d}$ as follows: 

We have two teams of people. The first team has $b$ people in it and they have $a$ apples in total, therefore equally sharing, each person in the first team gets $\frac{a}{b}$ apples. The second team has $d$ people in it and they have $c$ apples in total, therefore equally sharing, each person in the second team gets $\frac{c}{d}$ apples. If the two teams come together and share their apples between them, then each person in this bigger, combined team gets $\frac{a+c}{b+d}$ apples. It then makes sense that each person in this bigger, combined team would get no less than the smaller of the two ratios $\frac{c}{d}$ and would get no more than the larger of the two ratios $\frac{a}{b}$. Hence, $\frac{a+c}{b+d}$ should be in-between $\frac{c}{d}$ and $\frac{a}{b}$. More interesting is the claim that $\frac{a+c}{b+d}$ is neighbor fraction for each of the two original neighbor fractions.

\subsubsection*{Proof for $\frac{a+c}{b+d}$ is between $\frac{c}{d}$ and $\frac{a}{b}$}

Again, let's assume $\frac{c}{d} < \frac{a}{b}$, therefore from the statement of the problem we have $ad-bc = 1$. The argument that follows can easily be modified so it would hold even if the inequality is reversed. We shall make two comparisons: first between $\frac{a+c}{b+d}$ and $\frac{c}{d}$; and then between $\frac{a+c}{b+d}$ and $\frac{a}{b}$. The result of these two comparisons will give us the order among the three fractions showing that $\frac{a+c}{b+d}$ is between $\frac{c}{d}$ and $\frac{a}{b}$.

Let's compare $\frac{c}{d}$ and $\frac{a+c}{b+d}$:
\begin{displaymath}
\begin{split}
\frac{c}{d}\ \ &?\ \ \frac{a+c}{b+d}\\
     c(b+d)\ \ &?\ \ d(a+c)\\
      bc+cd\ \ &?\ \ ad+cd\\
         bc\ \ &?\ \ ad
\end{split}
\end{displaymath}
Since $ad-bc = 1$, we know $bc < ad$, therefore $\frac{c}{d} < \frac{a+c}{b+d}$.

Let's now compare $\frac{a+c}{b+d}$ and $\frac{a}{b}$:
\begin{displaymath}
\begin{split}
\frac{a+c}{b+d}\ \ &?\ \ \frac{a}{b}\\
         b(a+c)\ \ &?\ \ a(b+d)\\
          ab+bc\ \ &?\ \ ab+ad\\
             bc\ \ &?\ \ ad
\end{split}
\end{displaymath}
Since $ad-bc = 1$, we know $bc < ad$, therefore $\frac{a+c}{b+d} < \frac{a}{b}$.

Results of the two comparisons combined:
\[
	\frac{c}{d} < \frac{a+c}{b+d} < \frac{a}{b}\ \ \blacksquare
\]

\subsubsection*{Proof for $\frac{a+c}{b+d}$ is a neighbor fraction both for $\frac{a}{b}$ and $\frac{c}{d}$}
Let's assume $\frac{c}{d} < \frac{a}{b}$, thus $ad-bc = 1$. From the previous result, we then know $\frac{c}{d} < \frac{a+c}{b+d} < \frac{a}{b}$. If we can show that the numerator is $1$ for both of the differences, $\frac{a+c}{b+d}-\frac{c}{d}$ and $\frac{a}{b}-\frac{a+c}{b+d}$, we are done. 

For the first difference, the numerator is:
\begin{displaymath}
\begin{split}
(a+c) \cdot d - c \cdot (b+d) &= ad+cd - bc-cd\\
                              &= ad-bc\\
                              &= 1
\end{split}
\end{displaymath}
For the second difference, the numerator is:
\begin{displaymath}
\begin{split}
a \cdot (b+d) - b \cdot (a+c) &= ab+ad - ab-bc\\
                              &= ad-bc\\
                              &= 1\ \ \blacksquare
\end{split}
\end{displaymath}

\subsection*{Solution to Part c.}

\end{document}