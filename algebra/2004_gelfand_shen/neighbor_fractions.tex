\documentclass{article}

\usepackage{fouriernc}
\usepackage{enumerate} %Needed to have a, b, c instead of default 1, 2, 3 in enumerate environment%

\usepackage{amsmath}

\begin{document}

\section*{Problem-42}

\subsection*{Problem Statement}
Fractions $\frac{a}{b}$ and $\frac{c}{d}$ are called neighbor fractions if their difference $\frac{ad-bc}{bd}$ has numerator $\pm 1$, that is, $ad-bc = \pm 1$. Prove that 
\begin{enumerate}[a.]
\item in this case neither fraction can be simplified (that is, neither has any common factors in numerator and denominator);
\item if $\frac{a}{b}$ and $\frac{c}{d}$ are neighbor fractions, then $\frac{a+c}{b+d}$ is between them and is a neighbor fraction for both $\frac{a}{b}$ and $\frac{c}{d}$; moreover,
\item no fraction $\frac{e}{f}$ with positive integer $e$ and $f$ such that $f < b+d$ is between $\frac{a}{b}$ and $\frac{c}{d}$.

(In part b. of the statement the book says $\frac{a+b}{c+d}$ instead of $\frac{a+c}{b+d}$ and looks like that is a typo. For, neighbor fractions $\frac{1}{3}$ and $\frac{1}{2}$ the composite fraction $\frac{1+3}{1+2}$ is not in-between.)
\end{enumerate}
\subsection*{Solution to Part a.}
We assume that $a < b, c < d$ and if a common factor exists, it is greater than $1$. In other words, if the only common factor between say $a$ and $b$ is $1$, we consider them not having a common factor at all.

We shall use \textbf{proof by contradiction}, therefore we shall assume that two neighbor fractions share a common factor and that assumption will lead to a false conclusion, completing the proof. 

Say $a < b$, $a$ and $b$ share a common factor $f$, and $f$ is greater than $1$. Since the common factor is greater than $1$, $a > 1$. So, we have $b = f\cdot a$.

From the definition of neighbor fraction given in the problem statement, we have:

\begin{displaymath}
\begin{split}
	ad - bc         & = \pm 1\\
	ad-(f \cdot a)c & = \pm 1\\
	a(d-fc)         & = \pm 1\\
	d-fc            & = \pm \frac{1}{a}
\end{split}
\end{displaymath}
Since we started with integers on both sides of the equation and ended up with an integer on the left and a proper fraction on the right, we have a contradiction. Similar idea works if we assumed $a > b$ or if we assumed $c$ and $d$ also share a common factor. $\blacksquare$

\subsection*{Solution to Part b.}
Say $\frac{a}{b} > \frac{c}{d}$. Let's now interpret the neighbor fractions $\frac{a}{b}$ and $\frac{c}{d}$ as follows: 

We have two teams of people. The first team has $b$ people in it and they have $a$ apples in total, therefore equally sharing, each person in the first team gets $\frac{a}{b}$ apples. The second team has $d$ people in it and they have $c$ apples in total, therefore equally sharing, each person in the second team gets $\frac{c}{d}$ apples. If the two teams come together and share their apples between them, then each person in this bigger, combined team gets $\frac{a+c}{b+d}$ apples. It then makes sense that each person in this bigger, combined team would get no less than the smaller of the two ratios $\frac{c}{d}$ and would get no more than the larger of the two ratios $\frac{a}{b}$. Hence, $\frac{a+c}{b+d}$ should be in-between $\frac{c}{d}$ and $\frac{a}{b}$. More interesting is the claim that $\frac{a+c}{b+d}$ is neighbor fraction for each of the two original neighbor fractions.

\subsubsection*{Proof for $\frac{a+c}{b+d}$ is between $\frac{c}{d}$ and $\frac{a}{b}$}

Again, let's assume $\frac{c}{d} < \frac{a}{b}$, therefore from the statement of the problem we have $ad-bc = 1$. The argument that follows can easily be modified so it would hold even if the inequality is reversed. We shall make two comparisons: first between $\frac{a+c}{b+d}$ and $\frac{c}{d}$; and then between $\frac{a+c}{b+d}$ and $\frac{a}{b}$. The result of these two comparisons will give us the order among the three fractions showing that $\frac{a+c}{b+d}$ is between $\frac{c}{d}$ and $\frac{a}{b}$.

Let's compare $\frac{c}{d}$ and $\frac{a+c}{b+d}$:
\begin{displaymath}
\begin{split}
\frac{c}{d}\ \ &?\ \ \frac{a+c}{b+d}\\
     c(b+d)\ \ &?\ \ d(a+c)\\
      bc+cd\ \ &?\ \ ad+cd\\
         bc\ \ &?\ \ ad
\end{split}
\end{displaymath}
Since $ad-bc = 1$, we know $bc < ad$, therefore $\frac{c}{d} < \frac{a+c}{b+d}$.

Let's now compare $\frac{a+c}{b+d}$ and $\frac{a}{b}$:
\begin{displaymath}
\begin{split}
\frac{a+c}{b+d}\ \ &?\ \ \frac{a}{b}\\
         b(a+c)\ \ &?\ \ a(b+d)\\
          ab+bc\ \ &?\ \ ab+ad\\
             bc\ \ &?\ \ ad
\end{split}
\end{displaymath}
Since $ad-bc = 1$, we know $bc < ad$, therefore $\frac{a+c}{b+d} < \frac{a}{b}$.

Results of the two comparisons combined:
\[
	\frac{c}{d} < \frac{a+c}{b+d} < \frac{a}{b}\ \ \blacksquare
\]

\subsubsection*{Proof for $\frac{a+c}{b+d}$ is a neighbor fraction both for $\frac{a}{b}$ and $\frac{c}{d}$}
Let's assume $\frac{c}{d} < \frac{a}{b}$, thus $ad-bc = 1$. From the previous result, we then know $\frac{c}{d} < \frac{a+c}{b+d} < \frac{a}{b}$. If we can show that the numerator is $1$ for both of the differences, $\frac{a+c}{b+d}-\frac{c}{d}$ and $\frac{a}{b}-\frac{a+c}{b+d}$, we are done. 

For the first difference, the numerator is:
\begin{displaymath}
\begin{split}
(a+c) \cdot d - c \cdot (b+d) &= ad+cd - bc-cd\\
                              &= ad-bc\\
                              &= 1
\end{split}
\end{displaymath}
For the second difference, the numerator is:
\begin{displaymath}
\begin{split}
a \cdot (b+d) - b \cdot (a+c) &= ab+ad - ab-bc\\
                              &= ad-bc\\
                              &= 1\ \ \blacksquare
\end{split}
\end{displaymath}

\subsection*{Solution to Part c.}
Once again we assume that $\frac{c}{d} < \frac{a}{b}$, thus $ad-bc = 1$. Say $\frac{e}{f}$ is a fraction that lies between $\frac{c}{d}$ and $\frac{a}{b}$, therefore $\frac{c}{d} < \frac{e}{f} < \frac{a}{b}$. We thus have two inequalities: $af > be$ and $de > cf$. Since $a, b, c, d, e, f$ are all positive integers, we could rewrite the two inequalities as: $af-be \geq 1$ and $de-cf \geq 1$.

\begin{displaymath}
\begin{split}
	\frac{a}{b}-\frac{c}{d} &= \left( \frac{a}{b}-\frac{e}{f} \right) + \left( \frac{e}{f}-\frac{c}{d} \right)\\
	\frac{ad-bc}{bd}&= \frac{af-be}{bf}+\frac{de-cf}{df}\\
	\frac{1}{bd} &\geq \frac{1}{bf}+\frac{1}{df}\\
	\frac{1}{bd} &\geq \frac{b+d}{bdf}\\
	f &\geq b+d
\end{split}
\end{displaymath}
So, for any fraction $\frac{e}{f}$ that lies between the pair of neighbor fractions $\frac{c}{d}$ and $\frac{a}{b}$, $f$ cannot be less than $b+d$.  $\blacksquare$

In some sense $b+d$ is a hard limit on how low the denominator of such a fraction can get, signifying the closeness between the neighbor fractions themselves. In fact if we use the lowest value of $f$ therefore $f=b+d$ in the above inequality and go backwards we get $af-be = 1$ and $de-cf = 1$. We can then solve for $e$ as follows:
\begin{displaymath}
\begin{split}
	af-be &= de-cf\\
   (b+d)\cdot e &= (a+c)\cdot f\\
       e  &= a+c
\end{split}
\end{displaymath}
Seems like $\frac{a+c}{b+d}$ is the unique in-between fraction having the lowest possible denominator.

\section*{Problem-43}

\subsection*{Problem Statement}
A stick is divided by red marks into $7$ equal segments and by green marks into $13$ equal segments. Then it is cut into $20$ equal pieces. Prove that any piece (except the two end pieces) contain exactly one mark (which may be red or green).

\subsubsection{Proof for two end pieces do not contain any mark}
Say we are putting the marks from left to right. There would be $6$ red marks dividing the stick into $7$ equal segments. The $6$ red marks come at distances $\frac{1}{7}, \frac{2}{7},\ \ldots,\ \frac{5}{7}, \frac{6}{7}$ from left to right. Similarly there would be $12$ green marks dividing the stick into $13$ equal segments. The $12$ green marks come at distances $\frac{1}{13}, \frac{2}{13},\ \ldots,\ \frac{11}{13},\frac{12}{13}$ from left to right. Say we cut the stick from left to right at $19$ points and get $20$ equal pieces. The cuts happen at distances $\frac{1}{20}, \frac{2}{20},\ \ldots,\ \frac{18}{20}, \frac{19}{20}$ from left to right. At the left end we find $\frac{1}{20} < \frac{1}{13} < \frac{1}{7}$. So the left-end piece does not contain any mark. At the right end we find $\frac{6}{7} < \frac{12}{13} < \frac{19}{20}$. So, the right end-piece does not contain any mark either.  $\blacksquare$

\subsubsection{Proof for no two of the remaining $18$ pieces contain more than one mark}
Looking at the fractions $\frac{1}{13}$, $\frac{1}{20}$, and $\frac{1}{7}$ we see that $7+13=20$ looks similar to $b+d$ for the three fractions $\frac{c}{d}$, $\frac{a+c}{b+d}$, and $\frac{a}{b}$ from problem-42. So, there is an opportunity to use the property that $\frac{a+c}{b+d}$ comes in between $\frac{c}{d}$ and $\frac{a}{b}$. We also observe that for any two fractions $\frac{c}{d} < \frac{a}{b}$ with $a < b,\ c < d$, we have $\frac{c}{d} < \frac{a+c}{b+d} < \frac{a}{b}$; because, $c \cdot (b+d) < d \cdot (a+c)$ and $(a+c) \cdot b < (b+d) \cdot a$.

From the above obervations, we conclude that between any pair of red and green marks we always have a cut in-between; because, $\frac{c}{13} < \frac{a+c}{20} < \frac{a}{7}$. Similarly, between any pair of green and red marks we always have a cut in-between; because, $\frac{c'}{7} < \frac{a'+c'}{20} < \frac{a'}{13}$. Also, between two red marks or two green marks there always is an in-between cut; because the distance between these pairs are larger than $\frac{1}{20}$.   $\blacksquare$

Note, since the maximum value of $a$ is $6$ and the maximum value of $c$ is $12$, we have $a+c < 20$.

\subsubsection{Proof for each of the remaining $18$ pieces has at least one mark}
We need to put $6$ red and $12$ green, total $18$ marks. Since $\frac{1}{20} < \frac{1}{13} < 2 \cdot \frac{1}{20} < \frac{1}{7}$, the first red mark comes on the second piece from the left. If we now diregard the first two pieces and the last piece from the right-end, we are left with $20-3 = 17$ pieces to house the remaining $18-1=17$ marks. Since we showed earlier that no piece contain more than one mark, each of the $17$ remaining pieces must have exactly one mark to be able to house all $17$ of the remaining marks. Thus except for the two pieces at the two ends, each of the reamining $18$ pieces contain exactly one mark---either red or green.  $\blacksquare$

\end{document}












