\documentclass{article}

\usepackage{fouriernc}
\usepackage{amsmath}
\usepackage{amssymb}
\usepackage[shortlabels]{enumitem}

\begin{document}

\section*{Problem-304}
\subsection*{Problem Statement}
\begin{enumerate}[(a)]
	\item Find the side of a square having the same perimeter as a rectangle with sides $a$ and $b$.
	\item Find the side of a square having the same area as a rectangle with sides $a$ and $b$.
\end{enumerate}

\subsection*{Solution}
\begin{enumerate}[(a)]
	\item The question could also ask what are the sides of the rectangle having the same perimeter as a rectangle with sides $a$ and $b$, but has the maximum possible area. From \textbf{Problem-263}, we know that the rectangle with the maximum area would be a square.  Say the side of the square is $x$. Then we need $4x = 2(a+b)$, or $x = \frac{a+b}{2}$. Therefore, the side of the square is the arithmetic mean of the sides of the rectangle.
	\item The question could also ask what are the sides of the rectangle having the same area as a rectangle with sides $a$ and $b$, but has the minimum possible preimeter. From \textbf{Problem-264}, we know that the rectangle with the minimum perimeter would be a square. We need $x^2 = a \cdot b$, or $x = \sqrt{a \cdot b}$. Therefore, the side of the square is the geometric mean of the sides of the rectangle.
\end{enumerate}

\section*{Problem-317}
\subsection*{Problem Statement}
Prove the inequality between arithmetic and geometric means for $n = 4$.

\subsection*{Solution}
For non-negative integers $a,\ b,\ c,\ d$, we need to prove
\[
	\sqrt[4]{a \cdot b \cdot c \cdot d} \leq \frac{a+b+c+d}{4}
\]

We make the below two observations, $(1)$ and $(2)$ which we use during the proof.
\begin{equation*}
	\begin{aligned}
		a \cdot b\ \ \ \ \  &? \ \ \ \ \  \frac{a+b}{2} \cdot \frac{a+b}{2}\\
		4 \cdot a \cdot b\ \ \ \ \ &?\ \ \ \ \ (a+b)^2\\
		0 \ \ \ \ \  &? \ \ \ \ \  (a-b)^2\\
		0 \ \ \ \ \  &\leq \ \ \ \ \  (a-b)^2
	\end{aligned}
\end{equation*}
Thus we have 
\begin{equation}
	\frac{a+b}{2} \cdot \frac{a+b}{2} \geq a \cdot b
\end{equation}
Similarly,
\begin{equation*}
	\begin{aligned}
		\frac{a+b}{2} \cdot \frac{c+d}{2}\ \ \ \ \ &?\ \ \ \ \  \frac{a+b+c+d}{4} \cdot \frac{a+b+c+d}{4}
	\end{aligned}
\end{equation*}
Let $a+b = x$ and $c+d = y$, and we have
\begin{equation*}(a-b)^2
	\begin{aligned}
		\frac{x}{2} \cdot \frac{y}{2}\ \ \ \ \ &?\ \ \ \ \  \frac{x+y}{4} \cdot \frac{x+y}{4}\\
		4 \cdot x \cdot y\ \ \ \ \  &?\ \ \ \ \  (x+y)^2\\
		0\ \ \ \ \ &? \ \ \ \ \  (x-y)^2\\
		0\ \ \ \ \ &\leq \ \ \ \ \  (x-y)^2
	\end{aligned}
\end{equation*}
Thus we have,
\begin{equation}
	\frac{a+b+c+d}{4} \cdot \frac{a+b+c+d}{4} \geq \frac{a+b}{2} \cdot \frac{c+d}{2}
\end{equation}

Now we peform a sequence of transformations on the four numbers $(a, b, c, d)$. After each transformation, the sum remains $a+b+c+d$ but the product is bigger or equal to $a \cdot b \cdot c \cdot d$.
\[
	{(a, b, c, d) \mapsto \left(\frac{a+b}{2}, \frac{a+b}{2}, c, d \right)}
\]
In the above transformation, the product increases or remains the same (when $a = b$) because of $(1)$.
\[
	\left(\frac{a+b}{2}, \frac{a+b}{2}, c, d \right) \mapsto \left(\frac{a+b}{2}, \frac{a+b}{2}, \frac{c+d}{2}, \frac{c+d}{2} \right)
\]
In the above transformation, the product increases or remains the same (when $c = d$) because of $(1)$.
\[
	\left(\frac{a+b}{2}, \frac{a+b}{2}, \frac{c+d}{2}, \frac{c+d}{2} \right) \mapsto \left(\frac{a+b+c+d}{4}, \frac{a+b}{2}, \frac{a+b+c+d}{4}, \frac{c+d}{2} \right)
\]
In the above transformation, the product increases or remains the same (when $a+b = c+d$) because of $(2)$.
\[
	\left(\frac{a+b+c+d}{4}, \frac{a+b}{2}, \frac{a+b+c+d}{4}, \frac{c+d}{2} \right) \mapsto \left(\frac{a+b+c+d}{4}, \frac{a+b+c+d}{4}, \frac{a+b+c+d}{4}, \frac{a+b+c+d}{4} \right)
\]
In the above transformation, the product increases or remains the same (when $a+b = c+d$) because of $(2)$.
Let $\frac{a+b+c+d}{4} = S$. Then from the last transformation, we have
\begin{equation*}
	\begin{aligned}
		a \cdot b \cdot c \cdot d &\leq S \cdot S \cdot S \cdot S\\
		\sqrt[4]{a \cdot b \cdot c \cdot d} &\leq \frac{a+b+c+d}{4}\ \ \ \ \ \ \ \ \blacksquare
	\end{aligned}
\end{equation*}

\end{document}















