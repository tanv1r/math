\documentclass{article}

\usepackage{fouriernc}
\usepackage[shortlabels]{enumitem}

\begin{document}

\section*{Problem-304}
\subsection*{Problem Statement}
\begin{enumerate}[(a)]
	\item Find the side of a square having the same perimeter as a rectangle with sides $a$ and $b$.
	\item Find the side of a square having the same area as a rectangle with sides $a$ and $b$.
\end{enumerate}

\subsection*{Solution}
\begin{enumerate}[(a)]
	\item The question could also ask what are the sides of the rectangle having the same perimeter as a rectangle with sides $a$ and $b$, but has the maximum possible area. From \textbf{Problem-263}, we know that the rectangle with the maximum area would be a square.  Say the side of the square is $x$. Then we need $4x = 2(a+b)$, or $x = \frac{a+b}{2}$. Therefore, the side of the square is the arithmetic mean of the sides of the rectangle.
	\item The question could also ask what are the sides of the rectangle having the same area as a rectangle with sides $a$ and $b$, but has the minimum possible preimeter. From \textbf{Problem-264}, we know that the rectangle with the minimum perimeter would be a square. We need $x^2 = a \cdot b$, or $x = \sqrt{a \cdot b}$. Therefore, the side of the square is the geometric mean of the sides of the rectangle.
\end{enumerate}

\end{document}