\documentclass{article}

\usepackage{fouriernc}
\usepackage{amsmath}
\usepackage{amssymb}
\usepackage[shortlabels]{enumitem}

\begin{document}

\section*{Problem-304}
\subsection*{Problem Statement}
\begin{enumerate}[(a)]
	\item Find the side of a square having the same perimeter as a rectangle with sides $a$ and $b$.
	\item Find the side of a square having the same area as a rectangle with sides $a$ and $b$.
\end{enumerate}

\subsection*{Solution}
\begin{enumerate}[(a)]
	\item The question could also ask what are the sides of the rectangle having the same perimeter as a rectangle with sides $a$ and $b$, but has the maximum possible area. From \textbf{Problem-263}, we know that the rectangle with the maximum area would be a square.  Say the side of the square is $x$. Then we need $4x = 2(a+b)$, or $x = \frac{a+b}{2}$. Therefore, the side of the square is the arithmetic mean of the sides of the rectangle.
	\item The question could also ask what are the sides of the rectangle having the same area as a rectangle with sides $a$ and $b$, but has the minimum possible preimeter. From \textbf{Problem-264}, we know that the rectangle with the minimum perimeter would be a square. We need $x^2 = a \cdot b$, or $x = \sqrt{a \cdot b}$. Therefore, the side of the square is the geometric mean of the sides of the rectangle.
\end{enumerate}

\section*{Problem-317}
\subsection*{Problem Statement}
Prove the inequality between arithmetic and geometric means for $n = 4$.

\subsection*{Solution}
For non-negative integers $a,\ b,\ c,\ d$, we need to prove
\[
	\sqrt[4]{a \cdot b \cdot c \cdot d} \leq \frac{a+b+c+d}{4}
\]

We make the below two observations, $(1)$ and $(2)$ which we use during the proof.
\begin{equation*}
	\begin{aligned}
		a \cdot b\ \ \ \ \  &? \ \ \ \ \  \frac{a+b}{2} \cdot \frac{a+b}{2}\\
		4 \cdot a \cdot b\ \ \ \ \ &?\ \ \ \ \ (a+b)^2\\
		0 \ \ \ \ \  &? \ \ \ \ \  (a-b)^2\\
		0 \ \ \ \ \  &\leq \ \ \ \ \  (a-b)^2
	\end{aligned}
\end{equation*}
Thus we have 
\begin{equation}
	\frac{a+b}{2} \cdot \frac{a+b}{2} \geq a \cdot b
\end{equation}
Similarly,
\begin{equation*}
	\begin{aligned}
		\frac{a+b}{2} \cdot \frac{c+d}{2}\ \ \ \ \ &?\ \ \ \ \  \frac{a+b+c+d}{4} \cdot \frac{a+b+c+d}{4}
	\end{aligned}
\end{equation*}
Let $a+b = x$ and $c+d = y$, and we have
\begin{equation*}
	\begin{aligned}
		\frac{x}{2} \cdot \frac{y}{2}\ \ \ \ \ &?\ \ \ \ \  \frac{x+y}{4} \cdot \frac{x+y}{4}\\
		4 \cdot x \cdot y\ \ \ \ \  &?\ \ \ \ \  (x+y)^2\\
		0\ \ \ \ \ &? \ \ \ \ \  (x-y)^2\\
		0\ \ \ \ \ &\leq \ \ \ \ \  (x-y)^2
	\end{aligned}
\end{equation*}
Thus we have,
\begin{equation}
	\frac{a+b+c+d}{4} \cdot \frac{a+b+c+d}{4} \geq \frac{a+b}{2} \cdot \frac{c+d}{2}
\end{equation}

Now we peform a sequence of transformations on the four numbers $(a, b, c, d)$. After each transformation, the sum remains $a+b+c+d$ but the product is bigger or equal to $a \cdot b \cdot c \cdot d$.
\[
	{(a, b, c, d) \mapsto \left(\frac{a+b}{2}, \frac{a+b}{2}, c, d \right)}
\]
In the above transformation, the product increases or remains the same (when $a = b$) because of $(1)$.
\[
	\left(\frac{a+b}{2}, \frac{a+b}{2}, c, d \right) \mapsto \left(\frac{a+b}{2}, \frac{a+b}{2}, \frac{c+d}{2}, \frac{c+d}{2} \right)
\]
In the above transformation, the product increases or remains the same (when $c = d$) because of $(1)$.
\[
	\left(\frac{a+b}{2}, \frac{a+b}{2}, \frac{c+d}{2}, \frac{c+d}{2} \right) \mapsto \left(\frac{a+b+c+d}{4}, \frac{a+b}{2}, \frac{a+b+c+d}{4}, \frac{c+d}{2} \right)
\]
In the above transformation, the product increases or remains the same (when $a+b = c+d$) because of $(2)$.
\[
	\left(\frac{a+b+c+d}{4}, \frac{a+b}{2}, \frac{a+b+c+d}{4}, \frac{c+d}{2} \right) \mapsto \left(\frac{a+b+c+d}{4}, \frac{a+b+c+d}{4}, \frac{a+b+c+d}{4}, \frac{a+b+c+d}{4} \right)
\]
In the above transformation, the product increases or remains the same (when $a+b = c+d$) because of $(2)$.
Let $\frac{a+b+c+d}{4} = S$. Then from the last transformation, we have
\begin{equation*}
	\begin{aligned}
		a \cdot b \cdot c \cdot d &\leq S \cdot S \cdot S \cdot S\\
		\sqrt[4]{a \cdot b \cdot c \cdot d} &\leq \frac{a+b+c+d}{4}\ \ \ \ \ \ \ \ \blacksquare
	\end{aligned}
\end{equation*}

\section*{Problem-320}
\subsection*{Problem Statement}
Prove the inequality between arithmetic and geometric means for $n = 3$.

\subsection*{Solution}
We shall reduce the case for $n=3$ to the case for $n=4$ and use the result from \textbf{Problem-317} to finish it off.
\\

For three non-negative integers $a,\ b,\ c$ we are asked to prove
\[
	\sqrt[3]{a \cdot b \cdot c} \leq \frac{a+b+c}{3}
\]

We shall throw in the geometric mean of the three integers and form a group of four non-negative integers: $(a,b,c,\sqrt[3]{a \cdot b \cdot c})$. From \textbf{Problem-317} we know
\setcounter{equation}{0}
\begin{equation}
	\sqrt[4]{ abc\ \sqrt[3]{abc} } \leq \frac{a+b+c+\sqrt[3]{abc}}{4}
\end{equation}
We note that $\sqrt[4]{ abc\ \sqrt[3]{abc} } = \sqrt[4]{ (abc)^1 \cdot (abc)^{\frac{1}{3}} } = \sqrt[4]{ (abc)^{\frac{4}{3}} } = \sqrt[3]{abc}$. So, from $(1)$ now we have
\begin{equation*}
	\begin{aligned}
		\sqrt[3]{abc} &\leq \frac{a+b+c+\sqrt[3]{abc}}{4}\\
		4\sqrt[3]{abc} &\leq a+b+c+\sqrt[3]{abc}\\
		\sqrt[3]{a \cdot b \cdot c} &\leq \frac{a+b+c}{3}\ \ \ \ \ \ \ \ \ \blacksquare
	\end{aligned}
\end{equation*}

\section*{Problem-323}
\subsection*{Problem Statement}
Prove the inequality between arithmetic and geometric means for all integer $n \geq 2$.

\subsection*{Solution}
For $n \geq 2$ non-negative integers $a_1,\ a_2,\ \ldots,\ a_n$ we are asked to prove
\setcounter{equation}{0}
\begin{equation}
	\sqrt[n]{\prod_{k=1}^{n}\ a_k} \leq \frac{\sum_{k=1}^{n}\ a_k}{n}
\end{equation}
\subsubsection*{Proof-1}
We can prove $(1)$ for $n = 2^{m}$ where $m \geq 1$ using the  transformation idea from \textbf{Problem-317}. Say $n \geq 2$, lies in between $2^{p}$ and $2^{p+1}$. We already know $(1)$ holds for $2^{p+1}$ numbers; using that, we can use the idea from \textbf{Problem-320} to prove it for $2^{p+1}-1$ numbers as well. Applying the idea from \textbf{Problem-320} in sequence, starting wtih $2^{p+1}-1$ numbers and going backwards, we can prove $(1)$ for $n$.

\subsubsection*{Proof-2}
Let's scale each of the $n$ numbers $\sigma > 0$ times and see what happens to their arithmetic and geometric means. We start off with arithmetic mean:
\begin{equation*}
	\begin{aligned}
		&\frac{ \sum_{k=1}^{n}\ \sigma \cdot a_k }{n}\\
		&= \sigma \cdot \frac{\sum_{k=1}^n\ a_k}{n}
	\end{aligned}
\end{equation*}
Let's now look at the modified geometric mean:
\begin{equation*}
	\begin{aligned}
		&\sqrt[n]{ \prod_{k=1}^{n}\ \sigma \cdot a_k }\\
		&= \sqrt[n]{ \sigma^n \prod_{k=1}^{n}\ a_k }\\
		&= \sigma\ \sqrt[n]{ \prod_{k=1}{n}\ a_k }
	\end{aligned}
\end{equation*}
We see that both arithmetic and geometric means have been scaled by the same factor $\sigma$ thus $(1)$ holds for $a'_{k} = \sigma \cdot a_k$, if it holds for $a_k$.

Let $\sum_{k=1}^{n}\ a_k = \psi$. If all $a_k$'s are not zero, $\psi > 0$. We can now scale $a_k$'s to get $a'_k$'s such that $a'_k = \frac{n}{\psi} \cdot a_k$. Since scaling numbers by the same amount does not change the relation between their arithmetic and geometric means, if we can show $(1)$ for $a'_k$'s that would be sufficient. Now, observe that $\sum_{k=1}^{n}\ a'_k = n$. So, for $a'_k$'s the inequality $(1)$ takes the below form:
\[
	\sqrt[n]{ \prod_{k=1}^{n}\ a'_k }\ \  \leq\ \  1
\]
We shall now try to prove this derived inequality.
\begin{enumerate}
	\item For $n = 2$. We thus have $a'_1 + a'_2 = 2$. If both numbers are not equal to 1, we can let $a'_1 = 1-\delta$ and $a'_2 = 1+\delta$ with $\delta > 0$. Now the product $a'_1 \cdot a'_2 = (1-\delta)\ (1+\delta) = 1 - \delta^2 \leq 1$. That is what we needed.
	\item For $n=3$. Now we have $a'_1 + a'_2 + a'_3 = 3$. If all numbers are not equal to 1 (if they are, we are done), one should be less than 1 and another should be greater than 1. Say $a'_1 < 1$ and $a'_2 > 1$. So, $a'_1 - 1 < 0$ and $a'_2-1 > 0$.
	\begin{equation*}
		\begin{aligned}
			\left( a'_1 - 1 \right)\ \left( a'_2-1 \right) &< 0\\
			a'_1a'_2-a'_1-a'_2+1 &< 0\\
			a'_1a'_2+1 &< a'_1 + a'_2\\
			a'_1a'_2+1 + a'_3 &< a'_1 + a'_2 + a'_3\\
			a'_1a'_2+1 + a'_3 &< 3\\
			a'_1a'_2 + a'_3 &< 2
		\end{aligned}
	\end{equation*}
We have now back to the first case where we have two non-negative numbers $a'_1a'_2$ and $a'_3$ and they sum to 2. So, we know how to go about the proof henceforth.
	\item For $n=4$. We have $a'_1 + a'_2 + a'_3 + a'_4 = 4$.  With a similar argument to the second case, we arrive at the below inequality:
	\[
		a'_1a'_2 + a'_3 + a'_4 < 3
	\]
This takes us back to the second case and we know the rest.

For any $n > 3$, we can use the reducing idea in the second and third cases to go back to our base case, namely the case with $n=2$ which we have proved already.
\end{enumerate}

\subsubsection*{Proof-3}
Let's prove an identity. We start off with $(a+b+c)^3$:
\begin{equation*}
	\begin{aligned}
		&(a+b+c)^3\\
		&= a^3 + b^3 + c^3 + 3 a^2 b + 3 a b^2 + 3 b^2 c + 3 b c^2 + 3 c^2 a + 3 c a^2 + 6 abc\\
		&= a^3 + b^3 + c^3 - 3abc + 3 a^2 b + 3 a b^2 + 3 abc + 3 abc + 3 b^2 c + 3 b c^2 + 3 c a^2 + 3 abc + 3 c^2 a\\
		&= a^3 + b^3 + c^3 - 3abc + 3\ (a+b+c)\ (ab+bc+ca)
	\end{aligned}
\end{equation*}
We can now rearrange the two sides as follows:
\begin{equation*}
	\begin{aligned}
		&a^3 + b^3 + c^3 - 3abc\\
		&= (a+b+c)^3 - 3\ (a+b+c)\ (ab+bc+ca)\\
		&= (a+b+c) \left[ (a+b+c)^2 - 3(ab + bc + ca) \right]\\
		&= (a+b+c)\ \left( a^2+b^2+c^2 - ab - bc - ca \right)\\
		&= \frac{1}{2}\ (a+b+c) \left( 2a^2 + 2b^2 + 2c^2 - 2ab - 2bc - 2ca \right)\\
		&= \frac{1}{2}\ (a+b+c) \left[ \left(a^2-2ab+b^2\right) + \left( b^2 - 2bc + c^2 \right) + \left( c^2 - 2ca + a^2 \right) \right]\\
		&= \frac{1}{2}\ (a+b+c) \left[ (a-b)^2 + (b-c)^2 + (c-a)^2 \right]
	\end{aligned}
\end{equation*}

We see that for non-negative numbers $a,\ b,\ c$, the right side cannot be negative. So,
\begin{equation*}
	\begin{aligned}
		a^3 + b^3 + c^3 - 3abc &\geq 0\\
		abc \leq \frac{a^3+b^3+c^3}{3}
	\end{aligned}
\end{equation*}
Now if we make the replacements: $a = \sqrt[3]{p},\ b = \sqrt[3]{q},\ c = \sqrt[3]{r}$, then we have our required inequality between geometric and arithmetic means of three non-negative numbers:
\[
	\sqrt[3]{p \cdot q \cdot r} \leq \frac{p+q+r}{3}
\]
If we have more than 3 numbers for which we need to establish  $(1)$, we can always group (and replace the group with the group-sum) some of the numbers so that we end up with three numbers and appeal to our just-established fact.

\section*{Problem-326}
\subsection*{Problem Statement}
Assume that $a_1,\ \ldots,\ a_n$ are positive numbers. Prove that
\[
	\frac{a_1}{a_2} + \frac{a_2}{a_3} + \cdots + \frac{a_{n-1}}{a_n} + \frac{a_n}{a_1} \geq n
\]

\subsection*{Solution}
Conside the below product:
\[
	\frac{a_1}{a_2} \cdot \frac{a_2}{a_3} \cdot \cdots \cdot \frac{a_{n-1}}{a_n} \cdot \frac{a_n}{a_1}
\]
Each $a_i$ with $1 \leq i \leq n$ appears exactly once above and below, so the product evaluates to 1. We thus have:
\[
	\sqrt[n]{\frac{a_1}{a_2} \cdot \frac{a_2}{a_3} \cdot \cdots \cdot \frac{a_{n-1}}{a_n} \cdot \frac{a_n}{a_1}} = 1
\]
In other words, the geometric mean of the positive numbers $\frac{a_1}{a_2},\ \frac{a_2}{a_3},\ \ldots,\ \frac{a_{n-1}}{a_n},\ \frac{a_n}{a_1}$ is 1. Their arithmetic mean cannot be less than 1. So, we have:
\begin{equation*}
	\begin{aligned}
		\frac{\frac{a_1}{a_2} + \frac{a_2}{a_3} + \cdots + \frac{a_{n-1}}{a_n} + \frac{a_n}{a_1}}{n} &\geq 1\\
		\frac{a_1}{a_2} + \frac{a_2}{a_3} + \cdots + \frac{a_{n-1}}{a_n} + \frac{a_n}{a_1} &\geq n\ \ \ \ \ \blacksquare
	\end{aligned}
\end{equation*}

\section*{Problem-328}
\subsection*{Problem Statement}
Find the minimal value of $a+b$ if $a$ and $b$ are nonnegative numbers and $ab^2 = 1$.

\subsection*{Solution}
Since the arithmetic mean is at least as big as the geometric mean, we have $\frac{a+b}{2} \geq \sqrt{ab}$ or $a+b \geq 2\sqrt{ab}$; the equal case corresponds to the minimum value of $a+b$ and that happens when $a=b$. However, since $ab^2 = 1$, when $a+b$ assumes its minimum value, therefore, when $a=b$, we have $a^3=1$ or $a=1$. So, the minimum value of $a+b = 2$.

\section*{Problem-330}
\subsection*{Problem Statement}
Prove the inequality
\[
	\sqrt[3]{abc} \leq \frac{a+2b+3c}{3\sqrt[3]{6}}
\]

\subsection*{Solution}
We can rearrange the inequality as follows:
\begin{equation*}
	\begin{aligned}
		\sqrt[3]{6 \cdot abc} &\leq \frac{a+2b+3c}{3}\\
		\sqrt[3]{a \cdot 2b \cdot 3c} &\leq \frac{a + 2b + 3c}{3}
	\end{aligned}
\end{equation*}
So, the inequality in question is just a different form of the inequality for the geometric and arithmetic means of the three (nonnegative) numbers $a, 2b, 3c$.

\section*{Problem-331}
\subsection*{Problem Statement}
Prove that
\[
	\left( 1 + \frac{1}{10} \right)^{10} < \left( 1 + \frac{1}{11} \right)^{11}
\]

\subsection*{Solution}
We can consider the left side as the product of 11 numbers, one is 1 and all other equal to $1 + \frac{1}{10}$. We observe that the sum of these 11 numbers is $1 + 10\cdot  \left( 1 + \frac{1}{10} \right) = 12$. If we now consider the right side as the product of 11 numbers each equal to $1 + \frac{1}{11}$, we see that their sum is also 12. Therefore, considering the left and the right sides as the product of 11 numbers in the above sense, both sides have the same arithmetic mean. But the geometric mean, therefore the product, assumes the maximum value when all numbers are equal, which is the case with right side, not the left side. Hence the right side is bigger than the left.

\section*{Problem-332}
\subsection*{Problem Statement}
Prove that
\[
	\left( 1 + \frac{1}{10} \right)^{11} > \left( 1 + \frac{1}{11} \right)^{12}
\]

\subsection*{Solution}
The left side is the product of 11 numbers, each equal to $1 + \frac{1}{10}$. Their sum is $11 \cdot \left(1 + \frac{1}{10}\right) = 11+\frac{11}{10}$. Since, for a given arithmetic mean or sum, the geometric mean assumes its maximum when all number are equal, we can say that for the sum $11 + \frac{11}{10}$ of 11 numbers, the left side is already at the maximum possible product value.

Now, the right side can be considered as the product of 11 numbers as well: one equal to $\left(1+\frac{1}{11}\right)^2$ and the rest 10 all equal to $1 + \frac{1}{11}$. The sum of these 11 numbers is $10 \cdot \left(1 + \frac{1}{11}\right) + 1 + \frac{2}{11} + \frac{1}{11^2} = 11 + \frac{133}{121}$. Since, $\frac{133}{121} < \frac{11}{10}$, the right side has a smaller sum than the left side. So, the left side is the bigger among the two.

\end{document}















