\documentclass[12pt]{article}

\usepackage{fouriernc}
\usepackage{enumitem}
\usepackage{amsmath}

\begin{document}

\section*{Problem-90}
\subsection*{Problem Statement}
\begin{enumerate}[label=(\alph*)]
\item Multiply $(1+x)\ \left(1+x^2\right)$.
\item Multiply $(1+x)\ \left(1+x^2\right)\ \left(1+x^4\right)\ \left(1+x^8\right)$.
\item Compute $\left( 1 + x + x^2 + x^3 \right)^2$.
\item Compute $\left( 1+x+x^2+x^3+\ \ldots\  +x^9+x^{10}\right)^2$.
\item Find the coefficient of $x^{29}$ and $x^{30}$ in $\left( 1+x+x^2+x^3+\ \ldots\ +x^9+x^{10} \right)^3$.
\item Multiply $(1-x)\ \left( 1+x+x^2+x^3+\ \ldots\ +x^9+x^{10} \right)$.
\item Multiply $(a+b)\ \left( a^2-ab+b^2 \right)$.
\item Multiply $\left( 1-x+x^2-x^3+x^4-x^5+x^6-x^7+x^8-x^9+x^{10} \right)$ by\\ $\left( 1+x+x^2+x^3+x^4+x^5+x^6+x^7+x^8+x^9+x^{10} \right)$.
\end{enumerate}

\subsubsection*{Solution to Part-a}
\begin{equation*}
	\begin{array}{c}
		\phantom{\times99}x+1\\
		\underline{\times\phantom{99} x^2+1}\\
		\phantom{\times99}x+1\\
		\underline{\phantom\times x^3+x^2\phantom{999999999}}\\
		\phantom\times x^3+x^2+x+1\phantom{9999}
	\end{array}
\end{equation*}

\subsubsection*{Solution to Part-b}
From part-a we have $(1+x)\ \left(1+x^2\right) = 1+x+x^2+x^3$.
\begin{equation*}
	\begin{array}{c}
		\phantom{\times99}x^3+x^2+x+1\\
		\underline{\times\phantom{999999999} x^4+1}\\
		\phantom{\times99}x^3+x^2+x+1\\
		\underline{\phantom\times x^7+x^6+x^5+x^4\phantom{9999999999999999999999}}\\
		\phantom\times x^7+x^6+x^5+x^4+x^3+x^2+x+1\phantom{99999999999}
	\end{array}
\end{equation*}
Similar to the above, $\left(x^7+x^6+x^5+x^4+x^3+x^2+x+1\right)\ \left(1+x^8\right) = \sum_{n=0}^{15}x^n$.

\subsubsection*{Solution to Part-c}
\begin{equation*}
	\begin{array}{c}
		\phantom{\times99}x^3+x^2+x+1\\
		\underline{\times\phantom{99} x^3+x^2+x+1}\\
		\phantom{\times99}x^3+x^2+x+1\\
		\phantom\times x^4+x^3+x^2+x\phantom{9999}\\
		\phantom\times x^5+x^4+x^3+x^2\phantom{99999999999}\\
		\underline{\phantom\times x^6+x^5+x^4+x^3\phantom{999999999999999999}}\\
		\phantom\times x^6+2x^5+3x^4+4x^3+3x^2+2x+1\phantom{9999999}
	\end{array}
\end{equation*}
In regards to how many times they occur, we note that left and right of $x^3$ are mirror reflections. For example, there are as many $x^5$ as there are $x$. We may write the product as
\[
\sum_{n=0}^{2}(n+1) \cdot x^n\ +\ 4 \cdot x^3\ +\ \sum_{n=4}^{6} (6-n+1) \cdot x^n
\]

\subsubsection*{Solution to Part-d}
From the observation in part-c, the product here should be
\[
\sum_{n=0}^{9}(n+1) \cdot x^n\ +\ 11 \cdot x^{10}\ +\ \sum_{n=11}^{20}(20-n+1) \cdot x^n
\]
In other words the product is 
\[
x^{20}+2x^{19}+3x^{18}+\ \ldots\ +10x^{11}+11x^{10}+10x^9+9x^8+\ \ldots\ +3x^2+2x+1
\]

\subsubsection*{Solution to Part-e}
We observe that the coefficient of $x^{29}$ in $\left( 1+x+x^2+x^3+\ \ldots\ +x^9+x^{10} \right)^3$ is the count of ways we can get $29$ from adding three integers between $0$ and $10$ inclusive. In other words, it is the number of tuples $(a, b, c)$ where\\ $0 \leq a,b,c \leq 10$ and $a+b+c=29$. There are three such tuples: $(9, 10, 10)$, $(10, 9, 10)$, and $(10, 10, 9)$. So, the coefficient of $x^{29}$ is $3$. There is only one tuple whose elements sum to $30$, namely the tuple $(10, 10, 10)$; so the coefficient of $x^{30}$ is $1$.

Java code that follows, gives the below output for $\left( 1+x+x^2+x^3+\ \ldots\ +x^9+x^{10} \right)^3$ and we see that the coefficients of $x^{29}$ and $x^{30}$ are indeed $3$ and $1$ respectively. 

\begin{verbatim}
1  + 3 * x^1  + 6 * x^2  + 10 * x^3  + 15 * x^4  + 21 * x^5  + 28 * x^6  
+ 36 * x^7  + 45 * x^8  + 55 * x^9  + 66 * x^10  + 75 * x^11  + 82 * x^12  
+ 87 * x^13  + 90 * x^14  + 91 * x^15  + 90 * x^16  + 87 * x^17  
+ 82 * x^18  + 75 * x^19  + 66 * x^20  + 55 * x^21  + 45 * x^22  
+ 36 * x^23  + 28 * x^24  + 21 * x^25  + 15 * x^26  + 10 * x^27  
+ 6 * x^28  + 3 * x^29  + 1 * x^30
\end{verbatim}
\begin{verbatim}
  public static void main(String[] args) {
    int[] polynomial1 = new int[11];
    Arrays.fill(polynomial1, 1); 
    int[] polynomial2 = new int[11];
    Arrays.fill(polynomial2, 1);
    // we get the square of 1+x+x^2+...+x^10 here
    int[] square = multiply( polynomial1, polynomial2 );
    int[] cube = multiply( square, polynomial1 );
    printPolynomial(cube);
  }
  
  private static int[] multiply(int[] polynomial1, int[] polynomial2) {
    int m = polynomial1.length;
    int n = polynomial2.length;
    int[] product = new int[m+n];
    for (int i = 0; i < m; ++i) {
      for (int j = 0; j < n; ++j) {
        product[i+j] += polynomial1[i]*polynomial2[j];
      }
    }
    // get rid of extra zeros for the higher powers
    int k = m+n-1;
    while (k > 0 && product[k] == 0) {
      --k;
    }    
    int[] result = new int[k+1];
    for (int i = 0; i <= k; ++i) {
      result[i] = product[i];
    }   
    return result;
  }
  
  private static void printPolynomial(int[] poly) {
    System.out.println();
    System.out.print( String.format( "%d", poly[0] ) );
    for (int i = 1; i < poly.length; ++i) {
      System.out.print( String.format( "  + %d * x^%d", poly[i], i ) );
    }
    System.out.println();
  }
\end{verbatim}

\subsubsection*{Solution to Part-f}
\begin{displaymath}
\begin{split}
(1-x)\ \left( 1+x+x^2+x^3+\ \ldots\ +x^9+x^{10} \right) &= 1+\sum_{n=1}^{10}x^n-\sum_{n=1}^{10}x^n-x^{11}\\
														&= 1-x^{11}
\end{split}
\end{displaymath}

\end{document}