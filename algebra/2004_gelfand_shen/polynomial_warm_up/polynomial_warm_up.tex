\documentclass[12pt]{article}

\usepackage{fouriernc}
\usepackage{enumitem}
\usepackage{amsmath}

\begin{document}

\section*{Problem-90}
\subsection*{Problem Statement}
\begin{enumerate}[label=(\alph*)]
\item Multiply $(1+x)\ \left(1+x^2\right)$.
\item Multiply $(1+x)\ \left(1+x^2\right)\ \left(1+x^4\right)\ \left(1+x^8\right)$.
\item Compute $\left( 1 + x + x^2 + x^3 \right)^2$.
\item Compute $\left( 1+x+x^2+x^3+\ \ldots\  +x^9+x^{10}\right)^2$.
\item Find the coefficient of $x^{29}$ and $x^{30}$ in $\left( 1+x+x^2+x^3+\ \ldots\ +x^9+x^{10} \right)^3$.
\item Multiply $(1-x)\ \left( 1+x+x^2+x^3+\ \ldots\ +x^9+x^{10} \right)$.
\item Multiply $(a+b)\ \left( a^2-ab+b^2 \right)$.
\item Multiply $\left( 1-x+x^2-x^3+x^4-x^5+x^6-x^7+x^8-x^9+x^{10} \right)$ by\\ $\left( 1+x+x^2+x^3+x^4+x^5+x^6+x^7+x^8+x^9+x^{10} \right)$.
\end{enumerate}

\subsubsection*{Solution to Part-a}
\begin{equation*}
	\begin{array}{c}
		\phantom{\times99}x+1\\
		\underline{\times\phantom{99} x^2+1}\\
		\phantom{\times99}x+1\\
		\underline{\phantom\times x^3+x^2\phantom{999999999}}\\
		\phantom\times x^3+x^2+x+1\phantom{9999}
	\end{array}
\end{equation*}

\subsubsection*{Solution to Part-b}
From part-a we have $(1+x)\ \left(1+x^2\right) = 1+x+x^2+x^3$.
\begin{equation*}
	\begin{array}{c}
		\phantom{\times99}x^3+x^2+x+1\\
		\underline{\times\phantom{999999999} x^4+1}\\
		\phantom{\times99}x^3+x^2+x+1\\
		\underline{\phantom\times x^7+x^6+x^5+x^4\phantom{9999999999999999999999}}\\
		\phantom\times x^7+x^6+x^5+x^4+x^3+x^2+x+1\phantom{99999999999}
	\end{array}
\end{equation*}
Similar to the above, $\left(x^7+x^6+x^5+x^4+x^3+x^2+x+1\right)\ \left(1+x^8\right) = \sum_{n=0}^{15}x^n$.

\subsubsection*{Solution to Part-c}
\begin{equation*}
	\begin{array}{c}
		\phantom{\times99}x^3+x^2+x+1\\
		\underline{\times\phantom{99} x^3+x^2+x+1}\\
		\phantom{\times99}x^3+x^2+x+1\\
		\phantom\times x^4+x^3+x^2+x\phantom{9999}\\
		\phantom\times x^5+x^4+x^3+x^2\phantom{99999999999}\\
		\underline{\phantom\times x^6+x^5+x^4+x^3\phantom{999999999999999999}}\\
		\phantom\times x^6+2x^5+3x^4+4x^3+3x^2+2x+1\phantom{9999999}
	\end{array}
\end{equation*}
In regards to how many times they occur, we note that left and right of $x^3$ are mirror reflections. For example, there are as many $x^5$ as there are $x$. We may write the product as
\[
\sum_{n=0}^{2}(n+1) \cdot x^n\ +\ 4 \cdot x^3\ +\ \sum_{n=4}^{6} (6-n+1) \cdot x^n
\]

\subsubsection*{Solution to Part-d}
From the observation in part-c, the product here should be
\[
\sum_{n=0}^{9}(n+1) \cdot x^n\ +\ 11 \cdot x^{10}\ +\ \sum_{n=11}^{20}(20-n+1) \cdot x^n
\]
In other words the product is 
\[
x^{20}+2x^{19}+3x^{18}+\ \ldots\ +10x^{11}+11x^{10}+10x^9+9x^8+\ \ldots\ +3x^2+2x+1
\]
\end{document}