\documentclass[12pt]{article}

\usepackage{fouriernc}
\usepackage{amsmath}
\usepackage{halloweenmath}

\begin{document}

\section*{A degree 2 polynomial seems to have three distinct roots, how?}
Consider the below identity:
\[
\frac{ (x-a)\ (x-b) }{ (c-a)\ (c-b) } + \frac{ (x-a)\ (x-c) }{ (b-a)\ (b-c) } + \frac{ (x-b)\ (x-c) }{ (a-b)\ (a-c) } = 1
\]

We can rearrange:
\[
\frac{ (x-a)\ (x-b) }{ (c-a)\ (c-b) } + \frac{ (x-a)\ (x-c) }{ (b-a)\ (b-c) } + \frac{ (x-b)\ (x-c) }{ (a-b)\ (a-c) } - 1 = 0
\]
Now the left side of the above can be considered as a polynomial $P(x)$ with degree at most 2.
\[
P(x) = \frac{ (x-a)\ (x-b) }{ (c-a)\ (c-b) } + \frac{ (x-a)\ (x-c) }{ (b-a)\ (b-c) } + \frac{ (x-b)\ (x-c) }{ (a-b)\ (a-c) } - 1
\]
However, $a, b, c$ look like three distinct roots of $P(x)$. \textbf{ How?}

\section*{Problem-169}
Assume that $x_1,\ \ldots,\ x_{10}$ are different numbers, and $y_1,\ \ldots,\ y_{10}$ are arbitrary numbers. Prove that there is one and only one polynomial $P(x)$ of degree not exceeding 9 such that $P(x_1) = y_1,\ P(x_2) = y_2,\ \ldots,\ P(x_{10}) = y_{10}$.\\

\textbf{We get a set of 10 equations in 10 unknowns. As long as there is a solution, we have one polynomial. Uniqueness can be proved by reasoning about $\boldsymbol{R(x) = P(x)-Q(x)}$.}

\section*{Problem-222}
Imagine that now Achilles is running ten times more slowly than the turtle. When he comes to the place where the turtle was, it is at the distance ten times furthern than the initial one. When Achilles comes to that place, the turtle is far away---at the distance that is one hundred times further than the initial one, etc. So we come to the sum
\[
	1 + 10 + 100 +\ \ldots
\]
Of course, Achilles will never meet the turtle. But nevertheless we can substitute 10 for $q$ in the formula
\begin{equation}
	1 + q + q^2 + q^3 +\ \ldots = \frac{1}{1-q}.
\end{equation}
and get an (absurd) answer
\begin{equation}
	1 + 10 + 100 + 1000 +\ \ldots = \frac{1}{1-10} = -\frac{1}{9}.
\end{equation}
Is it possible to give a reasonable interpretation of the (absurd) statement ``Achilles will meet the turtle after running $-\frac{1}{9}$ meters''?\\

\textbf{Book hints that the answer is `Yes'. However, $(1)$ was derived assuming $0 < q < 1 \implies q^n = 0$ for sufficiently large $n$. Here $q = 10$. It does not seem like we could use $(1)$ and thus it seems like the derivation in $(2)$ is bogus; unless, we assume that as $10^n$ grows, at some point it reaches the cliff and then falls of to the ground and becomes zero! $\bigskull$}

\section*{Problem-243}
A cubic equation $x^3 + px^2 + qx +r = 0$ has three different roots $x_1,\ x_2,\ x_3$. Find
\[
	(x_1 - x_2)^2\ (x_2 - x_3)^2\ (x_1 - x_3)^2
\]
as an expression containing $p, q, r$.\\

\textbf{I think we should transform the given expression into another expression that contains only $(x_1 + x_2 + x_3)$ or $(x_1\ x_2 + x_2\ x_3 + x_3\ x_1)$ or $x_1\ x_2\ x_3$. Then by using Vieta's theorem for cubic equation we can express it in terms of $p,\ q,\ r$.}

\end{document}












