\documentclass[12pt]{article}

\usepackage{fouriernc}

\begin{document}

\section{A degree 2 polynomial seems to have three distinct roots, how?}
Consider the below identity:
\[
\frac{ (x-a)\ (x-b) }{ (c-a)\ (c-b) } + \frac{ (x-a)\ (x-c) }{ (b-a)\ (b-c) } + \frac{ (x-b)\ (x-c) }{ (a-b)\ (a-c) } = 1
\]

We can rearrange:
\[
\frac{ (x-a)\ (x-b) }{ (c-a)\ (c-b) } + \frac{ (x-a)\ (x-c) }{ (b-a)\ (b-c) } + \frac{ (x-b)\ (x-c) }{ (a-b)\ (a-c) } - 1 = 0
\]
Now the left side of the above can be considered as a polynomial $P(x)$ with degree at most 2.
\[
P(x) = \frac{ (x-a)\ (x-b) }{ (c-a)\ (c-b) } + \frac{ (x-a)\ (x-c) }{ (b-a)\ (b-c) } + \frac{ (x-b)\ (x-c) }{ (a-b)\ (a-c) } - 1
\]
However, $a, b, c$ look like three distinct roots of $P(x)$. \textbf{ How?}

\end{document}