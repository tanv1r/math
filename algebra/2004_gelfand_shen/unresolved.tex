\documentclass[12pt]{article}

\usepackage{fouriernc}
\usepackage{amsmath}

\begin{document}

\section*{A degree 2 polynomial seems to have three distinct roots, how?}
Consider the below identity:
\[
\frac{ (x-a)\ (x-b) }{ (c-a)\ (c-b) } + \frac{ (x-a)\ (x-c) }{ (b-a)\ (b-c) } + \frac{ (x-b)\ (x-c) }{ (a-b)\ (a-c) } = 1
\]

We can rearrange:
\[
\frac{ (x-a)\ (x-b) }{ (c-a)\ (c-b) } + \frac{ (x-a)\ (x-c) }{ (b-a)\ (b-c) } + \frac{ (x-b)\ (x-c) }{ (a-b)\ (a-c) } - 1 = 0
\]
Now the left side of the above can be considered as a polynomial $P(x)$ with degree at most 2.
\[
P(x) = \frac{ (x-a)\ (x-b) }{ (c-a)\ (c-b) } + \frac{ (x-a)\ (x-c) }{ (b-a)\ (b-c) } + \frac{ (x-b)\ (x-c) }{ (a-b)\ (a-c) } - 1
\]
However, $a, b, c$ look like three distinct roots of $P(x)$. \textbf{ How?}

\section*{Problem-169}
Assume that $x_1,\ \ldots,\ x_{10}$ are different numbers, and $y_1,\ \ldots,\ y_{10}$ are arbitrary numbers. Prove that there is one and only one polynomial $P(x)$ of degree not exceeding 9 such that $P(x_1) = y_1,\ P(x_2) = y_2,\ \ldots,\ P(x_{10}) = y_{10}$.\\

\textbf{We get a set of 10 equations in 10 unknowns. As long as there is a solution, we have one polynomial. Uniqueness can be proved by reasoning about $\boldsymbol{R(x) = P(x)-Q(x)}$.}

\end{document}