\documentclass[12pt]{article}

\usepackage{fouriernc}
\usepackage{amsmath}

\begin{document}
\section*{Problem-72}
\subsection*{Problem Statement}
There is a rule that allows us to square any number with the last digit $5$, namely, ``Drop this last digit out and get some $n$; multiply $n$ by $n+1$ and add the digits $2$ and $5$ to the end.'' For example, for $35^2$, we delete $5$ and get $3$, multiplying $3$ and $4$ we get $12$, adding ``$2$'' and ``$5$'' we get the answer: $1225$. Explain why this rule works.
\subsection*{Solution}
Let's say we want to find the square of an $n$ digits decimal number $d_{n-1}d_{n-2}\ldots d_{1}d_{0}$ with $d_{0}=5$. Following expansion justifies the rule:
\begin{displaymath}
\begin{split}
	{d_{n-1}d_{n-2}\ldots d_{1}5}^2 &= \left( d_{n-1}d_{n-2}\ldots d_{1}0 + 5 \right)^2\\
	                                &= {d_{n-1}d_{n-2}\ldots d_{1}0}^2 + 2\times d_{n-1}d_{n-2}\ldots d_{1}0 \times 5 + 25\\
	                                &= d_{n-1}d_{n-2}\ldots d_{1}0 \times \left(d_{n-1}d_{n-2}\ldots d_{1}0+10\right)+25\\
	                                &= d_{n-1}d_{n-2}\ldots d_{1}0 \times d_{n-1}d_{n-2}\ldots d_{1}'0+25\ \ \ \ \ [d_{1}' = d_{1}+1\ \textrm{with carry}]\\
	                                &= d_{n-1}d_{n-2}\ldots d_{1} \times d_{n-1}d_{n-2}\ldots d_{1}' \times 100 + 25
\end{split}
\end{displaymath}
Note, $d_{n-1}d_{n-2}\ldots d_{1}$ and $d_{n-1}d_{n-2}\ldots d_{1}'$ are consecutive integers.  $\blacksquare$
\end{document}